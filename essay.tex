% \documentclass[11pt,a4paper,sans]{moderncv}   % possible options include font size ('10pt', '11pt' and '12pt'), paper size ('a4paper', 'letterpaper', 'a5paper', 'legalpaper', 'executivepaper' and 'landscape') and font family ('sans' and 'roman')
\documentclass[12pt]{article}
\usepackage{hyperref}

% \usepackage[cm]{fullpage}
\usepackage[utf8]{inputenc}                   % 替换你正在使用的编码
\usepackage{CJKutf8}

% \usepackage[scale=0.75]{geometry}
\begin{CJK}{UTF8}{gkai}                       % 详情参阅CJK文件包
\title{关于中国近代史的思考}
\author{陈毅鸿}
\date{}
\begin{document}
\maketitle

\section{近代史的开端}
关于中国近代史的开端,学界有不同的说法。按照官方的说法,是起始于1840年第一次鸦片战争。
这场战争意味着外国帝国主义侵入中国的开端,此后的中国历史便主要是一部帝国主义侵华的历史\cite{yue:2008wk}。
另一种说法认为应当以明朝末期,传教士大量来华的那段时期作为中国近代史的起点。

徐中约先生认为这两种学派可以通过折衷的方式得到调和。即使把1840年界定为中国近代史的起点,我们也仍需熟悉
中国传统的国家和社会形态,以便了解后来中国对外界刺激所做的反应。

而黄仁宇先生则认为应该把1800年作为一个瞻前顾后的基点\cite{Huang:1997tj}。
1800年,清朝已达到成长的饱和点。「固定丁口,永不加税」,白莲教徒的反叛,为防止白银输出禁止鸦片进口等政策引导着一个新世纪的来临。
我更欣赏黄仁宇先生的观点,从宏观的视角去分析整个历史进程。避免了史料的庞杂与晦涩。

各个学派从不同的视角来定义中国近代史的开端,所说的都有道理。
但我们研究历史,并不是为了争论历史由何处开始,而是为了找寻历史要走向何处。
\section{清朝}
所谓「封建制度」的说法完全是为了应和马克思历史发展规律。
没有分封,哪来的封建?
而传统的教科书中对清朝的描述大多是脸谱化、妖魔化的。
对于历史人物的评价也大多有失偏颇。因本文篇幅有限,只简要描述几个重要的历史事件/阶段。
\subsection{虎门销烟}
一提起虎门销烟与鸦片战争,大多数国人都会联想起帝国主义、侵略、不平等条约等关键词。
而究竟发生了什么反倒没有什么人去关心。茅海建先生在《天朝的崩溃》中对整个鸦片战争进行了深入详细地研究与分析。
很多事情甚至荒唐的令人无法相信,类似
「夷兵除枪炮外,击刺俱非所娴,而其腿足裹缠,结束紧密,屈伸皆所不便,若至岸上更无能为,是其强非不可制也」等言行不在此一一赘述。

我们需要反思的是为什么没能从中吸取教训,甚至时至今日在号称人民民主的共和国中依然有民族主义小报到处贩卖民粹与仇恨。
在鸦片战争前后,特别有意思的地方是整个清廷都没有人了解外部世界,哪怕基本的常识都缺乏。到了第二次鸦片战争,依然如此。
已经签署的条约不予履行、三番五次地阻止修约。即使再次战败,仍然不愿去了解敌人。而同一时期,日本却正在进行明治维新。
三十多年后战胜了清廷,再之后战争了俄国。而那时中国依然在进行着改良与革命的赛跑。
\subsection{戊戌变法}
谈到戊戌变法,就不得不说茅海建先生的《戊戌变法史事考》。
因为以往关于戊戌变法的记载大都出自康梁之手,而茅海建先生通过详细地考证史料。
给我们描述了维新变法演化的路径,纠正了很多对戊戌变法错误的理解。

衣带诏、谭嗣同伪诗等事更是证明了我一贯坚持的观点:
须知正义的事业是通过正义的手段实现的。

另外一件值得人深思的事是:「慈禧太后向京中各衙门堂官公布光绪帝的病情」。
一百多年之后的今日,重大政治事件仍然用这种上不得台面的手段来推进。
不得不说是这个国家的悲哀。

% \subsection{太平天国与义和团}
% 太平天国与义和团是两次灾难性事件。
% 它们集中体现了中国政治中
\subsection{清末新政}
关于清末新政,近年来很流行一个说法:「不改革等死,改革找死」。
但这种说法明显是对清末新政的误读。

我们可以把清末新政分为两个阶段,慈禧去世前是第一个阶段。
在这个阶段,可以说是改革与革命在赛跑,而且改革还占有优势。
这个阶段革命人士反而有点像恐怖分子,谁提倡改革反而敌视谁。
甚至刺杀出国考察的五大臣。
而且在这个时期,改革是受到士绅、商人、各地实力派支持的。

第二个阶段是光绪、慈禧去世之后,载沣任监国摄政王直到辛亥革命。
在这个阶段,清廷不断地把改革的支持者推到革命者的阵营。
把应当放权的改革变成到处集权的改革,至此改革的失败已经是可以遇见的。
载沣罢黜袁世凯,把军权握在皇室手中。
但他们并不知道拥有权力并不重要,能驾驭权力才重要。
辛亥革命发生之后,他们虽然有军权,却不能指挥动军队。
尤其是当皇族内阁出炉之后,全国哗然,人心尽失。
\section{中华民国}
从1911年到1949年的这段历史基本可以分为三个阶段:革命与和谈(1911-1916),北洋时期(1916-1927),党国(1927-1949)。
\subsection{革命与和谈}
我认为这个时期是中国政治最美好的时期:开放报禁,党禁、
流血很少的革命、体面退位的皇族、革命派与实力派的妥协、
共和制度的建立。
\subsection{北洋时期}
北洋军阀一直受到批评。但认真研究这段历史我们可以发现,
无论是内政外交,北洋政府都做出了有成效的努力。
而所谓的军阀混战,并没有造成过多的伤亡。
当时的政治家还是尊重共和体制的。
\subsection{党国}
略
\section{「 新」中国}
略
\section{当代政治文明}
时至今日,我们对于政治文明的理解依然如此浅薄。
百年之前关于民主与民众素质的争论,依然没有结果。
反倒是类似的争论越来越少,不是因为已经取得共识,而是被禁止。
正如萨托利所言,对于民主、自由的理解,后人未必强于前人\cite{satuoli:1993uk}。
从当今中国来看,后人的智慧甚至远远比不上前人。

纵观中国近代史,政治文明的底线不断被刷新。
清廷虽然有很大的局限性,诸如不遵守自己签署的条约、不遵守战争法则、不尊重私有财产。
但这些是由于当时历史的局限性造成的。毕竟皇权社会不但有一套明规则,亦有一套行之有效的潜规则。
整个清廷对于内部事物的处理大体上还是尊重这套游戏规则的。
虽然在某些重要事件上会用一些非常规的手段,但皇室的尊严还是限制着太出格的行为。

至于辛亥之后至北洋政府这个阶段,共和制度刚建立起来之时,各方至少在表面上是遵守共和这一套规则的。
至少选举制度的程序性是被人尊重的。
即使曹锟通过贿选当上大总统,但也说明他是要通过选举的程序获得执政的合法性。
但从另一方面来讲,各方又太缺乏政治智慧。革命派在自己人当总统时就要总统制,袁世凯当总统了就要总理制。
另外,只要袁世凯令清廷退位即支持袁世凯任中华民国大总统一事也反映出革命党人太不把选举当一回事了。
再者,袁世凯就任大总统之后,各种事情都会受到议会规则的掣肘。
议员们缺乏妥协的精神与手段,但一旦袁世凯使用一些不光彩的手段,议员们就立刻妥协了。
而孙中山带领的二次革命更是令人不齿,打着革命的名义进行武装叛变。
至于后来用二十一条来打击袁世凯,更是一种龌龊的手段来进行政治斗争。
但在这个阶段,议会政治、选举制度还是受到尊重的,政治斗争也只是止于夺取权力。
失败者下野之后还是会受到一定的照顾。

之后的党国时期,列宁式的政党掌权。
而列宁式政党的一大特点即下级服从上级,中央服从一人或多人。
国民党的基本方针是「将党放在国上」(孙文),直接结果就是议会完全形式化。
从蒋介石依靠廖仲恺之死上位就可以看出,这个时期政治斗争已经到了必须把对手从肉体上消灭的地步。
这个时期大多数的时间处于战争状态,很多事是不得已而为之。
但总的来看,政治文明的底线是越来越低。

「新中国」政治斗争的手法是始于延安时期的。
其手段之残忍、手法之卑劣、牵连范围之广,
高华先生在《红太阳是怎样升起的——延安整风的来龙去脉》中已经有详细的描述。
中共建政之后的政治斗争基本上是相同的套路。
至于所谓的民主,早就被专政吞噬的一干二净。
从打AB团开始,延安整风,土改,
镇压反革命,三反五反,社会性主义改造,肃清反革命,
反右派,反右倾,大跃进,文化大革命,
反资产阶级自由化,六四天安门屠杀。

到底历史要走向何方?

\clearpage
\bibliographystyle{plain}
\bibliography{history}
\end{CJK}
\end{document}
